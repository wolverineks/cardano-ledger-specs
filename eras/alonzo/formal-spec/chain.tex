\section{Blockchain layer}
\label{sec:chain}

\newcommand{\Proof}{\type{Proof}}
\newcommand{\Seedl}{\mathsf{Seed}_\ell}
\newcommand{\Seede}{\mathsf{Seed}_\eta}
\newcommand{\activeSlotCoeff}[1]{\fun{activeSlotCoeff}~ \var{#1}}
\newcommand{\slotToSeed}[1]{\fun{slotToSeed}~ \var{#1}}
\newcommand{\isOverlaySlot}[3]{\fun{isOverlaySlot}~\var{#1}~\var{#2}~\var{#3}}

\newcommand{\T}{\type{T}}
\newcommand{\vrf}[3]{\fun{vrf}_{#1} ~ #2 ~ #3}
\newcommand{\verifyVrf}[4]{\fun{verifyVrf}_{#1} ~ #2 ~ #3 ~#4}

\newcommand{\HashHeader}{\type{HashHeader}}
\newcommand{\HashBBody}{\type{HashBBody}}
\newcommand{\bhHash}[1]{\fun{bhHash}~ \var{#1}}
\newcommand{\bHeaderSize}[1]{\fun{bHeaderSize}~ \var{#1}}
\newcommand{\bSize}[1]{\fun{bSize}~ \var{#1}}
\newcommand{\bBodySize}[1]{\fun{bBodySize}~ \var{#1}}
\newcommand{\OCert}{\type{OCert}}
\newcommand{\BHeader}{\type{BHeader}}
\newcommand{\BHBody}{\type{BHBody}}

\newcommand{\bheader}[1]{\fun{bheader}~\var{#1}}
\newcommand{\hsig}[1]{\fun{hsig}~\var{#1}}
\newcommand{\bprev}[1]{\fun{bprev}~\var{#1}}
\newcommand{\bhash}[1]{\fun{bhash}~\var{#1}}
\newcommand{\bvkcold}[1]{\fun{bvkcold}~\var{#1}}
\newcommand{\bseedl}[1]{\fun{bseed}_{\ell}~\var{#1}}
\newcommand{\bprfn}[1]{\fun{bprf}_{n}~\var{#1}}
\newcommand{\bseedn}[1]{\fun{bseed}_{n}~\var{#1}}
\newcommand{\bprfl}[1]{\fun{bprf}_{\ell}~\var{#1}}
\newcommand{\bocert}[1]{\fun{bocert}~\var{#1}}
\newcommand{\bnonce}[1]{\fun{bnonce}~\var{#1}}
\newcommand{\bleader}[1]{\fun{bleader}~\var{#1}}
\newcommand{\hBbsize}[1]{\fun{hBbsize}~\var{#1}}
\newcommand{\bbodyhash}[1]{\fun{bbodyhash}~\var{#1}}
\newcommand{\overlaySchedule}[4]{\fun{overlaySchedule}~\var{#1}~\var{#2}~{#3}~\var{#4}}

\newcommand{\PrtclState}{\type{PrtclState}}
\newcommand{\PrtclEnv}{\type{PrtclEnv}}
\newcommand{\OverlayEnv}{\type{OverlayEnv}}
\newcommand{\VRFState}{\type{VRFState}}
\newcommand{\NewEpochEnv}{\type{NewEpochEnv}}
\newcommand{\NewEpochState}{\type{NewEpochState}}
\newcommand{\PoolDistr}{\type{PoolDistr}}
\newcommand{\BBodyEnv}{\type{BBodyEnv}}
\newcommand{\BBodyState}{\type{BBodyState}}
\newcommand{\RUpdEnv}{\type{RUpdEnv}}
\newcommand{\ChainEnv}{\type{ChainEnv}}
\newcommand{\ChainState}{\type{ChainState}}
\newcommand{\ChainSig}{\type{ChainSig}}



\subsection{Block Body Transition}
\label{sec:block-body-trans}


Figure~\ref{fig:rules:bbody} includes an additional check that the sum of the
execution units of all transactions in a block does not exceed the maximum total units per
block (specified in a protocol parameter).

\begin{figure}[ht]
  \begin{equation}\label{eq:bbody}
    \inference[Block-Body]
    {
      \var{txs} \leteq \bbody{block}
      &
      \var{bhb} \leteq \bhbody{(\bheader{block})}
      &
      \var{hk} \leteq \hashKey{(\bvkcold{bhb})}
      \\~\\
      \bBodySize{txs} = \hBbsize{bhb}
      &
      \fun{bbodyhash}~{txs} = \bhash{bhb}
      \\~\\
      \\
      \var{slot} \leteq \bslot{bhb}
      &
      \var{fSlot} \leteq \fun{firstSlot}~(\epoch{slot})
      \\~\\
      \hldiff{\sum_{tx\in txs} \fun{totExunits}~{tx} \leq \fun{maxBlockExUnits}~\var{pp}}
      \\~\\
      {
        {\begin{array}{c}
                 \bslot{bhb} \\
                 \var{pp} \\
                 \var{acnt}
        \end{array}}
        \vdash
             \var{ls} \\
        \trans{\hyperref[fig:rules:ledger-sequence]{ledgers}}{\var{txs}}
             \var{ls}' \\
      }
    }
    {
      {\begin{array}{c}
               \var{pp} \\
               \var{acnt}
      \end{array}}
      \vdash
      {\left(\begin{array}{c}
            \var{ls} \\
            \var{b} \\
      \end{array}\right)}
      \trans{bbody}{\var{block}}
      {\left(\begin{array}{c}
            \varUpdate{\var{ls}'} \\
            \varUpdate{\fun{incrBlocks}~{\left(\isOverlaySlot{fSlot}{(\fun{d}~pp)}{slot}\right)}~{hk}~{b}} \\
      \end{array}\right)}
    }
  \end{equation}
  \caption{BBody rules}
  \label{fig:rules:bbody}
\end{figure}

We have also defined a function that transforms the Shelley ledger state into
the corresponding Goguen one, see Figure~\ref{fig:functions:to-shelley}. We
refer to the Shelley-era protocol parameter type as $\ShelleyPParams$, and the corresponding Goguen
type as $\PParams$. We use the notation $\var{chainstate}_{x}$ to represent
variable $x$ in the chain state. We do not specify the variables that
remain unchanged during the transition.

%%
%% Figure - Shelley to Goguen Transition
%%
\begin{figure}[htb]
  \emph{Types and Constants}
  %
  \begin{align*}
      & \NewParams ~=~ (\Language \mapsto \CostMod) \times \Prices \times \ExUnits \times \ExUnits \\
      & ~~~~~~~~ \times \N \times \Coin \times \N \times \N \\
      & \text{the type of new parameters to add for Goguen}
      \nextdef
      & \var{ivPP} ~\in~ \NewParams \\
      & \text{the initial values for new Goguen parameters}
  \end{align*}
  \emph{Shelley to Goguen Transition Functions}
  %
  \begin{align*}
      & \fun{toGoguen} \in \ShelleyChainState \to \ChainState \\
      & \fun{toGoguen}~\var{chainstate} =~\var{chainstate'} \\
      &~~\where \\
      &~~~~\var{chainstate'}_{pparams}~=~(\var{pp}\cup \var{ivPP})~ - ~\var{minUTxOValue}\\
      & \text{transform Shelley chain state to Goguen state}
  \end{align*}
  \caption{Shelley to Goguen State Transtition}
  \label{fig:functions:to-shelley}
\end{figure}
